\begin{pages}
    \begin{Rightside}
    \selectlanguage{greek}
        \beginnumbering
		\pstart[
				\chapter{Ἰωάννης ἐν τῇ Πάτμῳ}
				\markboth{John on the Isle of Patmos}
				]
		\renewcommand{\LettrineFontHook}{\PHtitl}
		\lettrine[lines=3]{Τ}{εκνία} μου, ταῦτα γράφω ὑμῖν ἵνα μὴ ἁμάρτητε. καὶ ἐάν τις ἁμάρτῃ, Παράκλητον ἔχομεν πρὸς τὸν Πατέρα Ἰησοῦν Χριστὸν δίκαιον· καὶ αὐτὸς ἱλασμός ἐστιν περὶ τῶν ἁμαρτιῶν ἡμῶν, οὐ περὶ τῶν ἡμετέρων δὲ μόνον ἀλλὰ καὶ περὶ ὅλου τοῦ κόσμου. καὶ ἐν τούτῳ γινώσκομεν ὅτι ἐγνώκαμεν αὐτόν, ἐὰν τὰς ἐντολὰς αὐτοῦ τηρῶμεν. ὁ λέγων ὅτι Ἔγνωκα αὐτόν, καὶ τὰς ἐντολὰς αὐτοῦ μὴ τηρῶν, ψεύστης ἐστίν, καὶ ἐν τούτῳ ἡ ἀλήθεια οὐκ ἔστιν· ὃς δ’ ἂν τηρῇ αὐτοῦ τὸν λόγον ἀληθῶς ἐν τούτῳ ἡ ἀγάπη τοῦ Θεοῦ τετελείωται. Ἐν τούτῳ γινώσκομεν ὅτι ἐν αὐτῷ ἐσμεν· ὁ λέγων ἐν αὐτῷ μένειν ὀφείλει καθὼς ἐκεῖνος περιεπάτησεν καὶ αὐτὸς οὕτως περιπατεῖν.
		\pend
		\pstart
		Ἀγαπητοί, οὐκ ἐντολὴν καινὴν γράφω ὑμῖν, ἀλλ’ ἐντολὴν παλαιὰν ἣν εἴχετε ἀπ’ ἀρχῆς· ἡ ἐντολὴ ἡ παλαιά ἐστιν ὁ λόγος ὃν ἠκούσατε. πάλιν ἐντολὴν καινὴν γράφω ὑμῖν, ὅ ἐστιν ἀληθὲς ἐν αὐτῷ καὶ ἐν ὑμῖν, ὅτι ἡ σκοτία παράγεται καὶ τὸ φῶς τὸ ἀληθινὸν ἤδη φαίνει. Ὁ λέγων ἐν τῷ φωτὶ εἶναι καὶ τὸν ἀδελφὸν αὐτοῦ μισῶν ἐν τῇ σκοτίᾳ ἐστὶν ἕως ἄρτι. ὁ ἀγαπῶν τὸν ἀδελφὸν αὐτοῦ ἐν τῷ φωτὶ μένει, καὶ σκάνδαλον ἐν αὐτῷ οὐκ ἔστιν· ὁ δὲ μισῶν τὸν ἀδελφὸν αὐτοῦ ἐν τῇ σκοτίᾳ ἐστὶν καὶ ἐν τῇ σκοτίᾳ περιπατεῖ, καὶ οὐκ οἶδεν ποῦ ὑπάγει, ὅτι ἡ σκοτία ἐτύφλωσεν τοὺς ὀφθαλμοὺς αὐτοῦ. Γράφω ὑμῖν, τεκνία, ὅτι ἀφέωνται ὑμῖν αἱ ἁμαρτίαι διὰ τὸ ὄνομα αὐτοῦ. γράφω ὑμῖν, πατέρες, ὅτι ἐγνώκατε τὸν ἀπ’ ἀρχῆς· γράφω ὑμῖν, νεανίσκοι, ὅτι νενικήκατε τὸν πονηρόν. ἔγραψα ὑμῖν, παιδία, ὅτι ἐγνώκατε τὸν Πατέρα. ἔγραψα ὑμῖν, πατέρες, ὅτι ἐγνώκατε τὸν ἀπ’ ἀρχῆς. ἔγραψα ὑμῖν, νεανίσκοι, ὅτι ἰσχυροί ἐστε καὶ ὁ λόγος τοῦ Θεοῦ ἐν ὑμῖν μένει καὶ νενικήκατε τὸν πονηρόν. Μὴ ἀγαπᾶτε τὸν κόσμον μηδὲ τὰ ἐν τῷ κόσμῳ. ἐάν τις ἀγαπᾷ τὸν κόσμον, οὐκ ἔστιν ἡ ἀγάπη τοῦ Πατρὸς ἐν αὐτῷ· ὅτι πᾶν τὸ ἐν τῷ κόσμῳ, ἡ ἐπιθυμία τῆς σαρκὸς καὶ ἡ ἐπιθυμία τῶν ὀφθαλμῶν καὶ ἡ ἀλαζονία τοῦ βίου, οὐκ ἔστιν ἐκ τοῦ πατρός, ἀλλὰ ἐκ τοῦ κόσμου ἐστίν. καὶ ὁ κόσμος παράγεται καὶ ἡ ἐπιθυμία αὐτοῦ· ὁ δὲ ποιῶν τὸ θέλημα τοῦ Θεοῦ μένει εἰς τὸν αἰῶνα.
		\pend
		\pstart
	 	Παιδία, ἐσχάτη ὥρα ἐστίν, καὶ καθὼς ἠκούσατε ὅτι ἀντίχριστος ἔρχεται, καὶ νῦν ἀντίχριστοι πολλοὶ γεγόνασιν· ὅθεν γινώσκομεν ὅτι ἐσχάτη ὥρα ἐστίν. ἐξ ἡμῶν ἐξῆλθαν, ἀλλ’ οὐκ ἦσαν ἐξ ἡμῶν· εἰ γὰρ ἐξ ἡμῶν ἦσαν, μεμενήκεισαν ἂν μεθ’ ἡμῶν· ἀλλ’ ἵνα φανερωθῶσιν ὅτι οὐκ εἰσὶν πάντες ἐξ ἡμῶν. καὶ ὑμεῖς χρῖσμα ἔχετε ἀπὸ τοῦ Ἁγίου, καὶ οἴδατε πάντες. οὐκ ἔγραψα ὑμῖν ὅτι οὐκ οἴδατε τὴν ἀλήθειαν, ἀλλ’ ὅτι οἴδατε αὐτήν, καὶ ὅτι πᾶν ψεῦδος ἐκ τῆς ἀληθείας οὐκ ἔστιν. Τίς ἐστιν ὁ ψεύστης εἰ μὴ ὁ ἀρνούμενος ὅτι Ἰησοῦς οὐκ ἔστιν ὁ Χριστός; οὗτός ἐστιν ὁ ἀντίχριστος, ὁ ἀρνούμενος τὸν Πατέρα καὶ τὸν Υἱόν. πᾶς ὁ ἀρνούμενος τὸν Υἱὸν οὐδὲ τὸν Πατέρα ἔχει· ὁ ὁμολογῶν τὸν Υἱὸν καὶ τὸν Πατέρα ἔχει. ὑμεῖς ὃ ἠκούσατε ἀπ’ ἀρχῆς, ἐν ὑμῖν μενέτω. ἐὰν ἐν ὑμῖν μείνῃ ὃ ἀπ’ ἀρχῆς ἠκούσατε, καὶ ὑμεῖς ἐν τῷ Υἱῷ καὶ ἐν τῷ Πατρὶ μενεῖτε. καὶ αὕτη ἐστὶν ἡ ἐπαγγελία ἣν αὐτὸς ἐπηγγείλατο ἡμῖν, τὴν ζωὴν τὴν αἰώνιον. Ταῦτα ἔγραψα ὑμῖν περὶ τῶν πλανώντων ὑμᾶς. καὶ ὑμεῖς τὸ χρῖσμα ὃ ἐλάβετε ἀπ’ αὐτοῦ μένει ἐν ὑμῖν, καὶ οὐ χρείαν ἔχετε ἵνα τις διδάσκῃ ὑμᾶς· ἀλλ’ ὡς τὸ αὐτοῦ χρῖσμα διδάσκει ὑμᾶς περὶ πάντων, καὶ ἀληθές ἐστιν καὶ οὐκ ἔστιν ψεῦδος, καὶ καθὼς ἐδίδαξεν ὑμᾶς, μένετε ἐν αὐτῷ. Καὶ νῦν, τεκνία, μένετε ἐν αὐτῷ, ἵνα ἐὰν φανερωθῇ σχῶμεν παρρησίαν καὶ μὴ αἰσχυνθῶμεν ἀπ’ αὐτοῦ ἐν τῇ παρουσίᾳ αὐτοῦ. ἐὰν εἰδῆτε ὅτι δίκαιός ἐστιν, γινώσκετε ὅτι καὶ πᾶς ὁ ποιῶν τὴν δικαιοσύνην ἐξ αὐτοῦ γεγέννηται.
		\pend
        \endnumbering
    \end{Rightside}
    \begin{Leftside}
        \beginnumbering
        \pstart[
        			\chapter{John on the Isle of Patmos}
        			]
        	\renewcommand\LettrineFontHook{\Zallmanfamily}
		\lettrine[lines=3]{T}{he} Revelation of Jesus Christ, which God gave Him to show His servants what must soon happen; and He made it known through the sending of His messenger to His servant John, who confirms everything that he saw, namely the word of God and the testimony of Jesus Christ. Blessed is the reader and the people who listen to the words of the prophecy and (blessed is) the one who heeds what is written in it (the prophecy), for the time is near.
		\pend
        \endnumbering
    \end{Leftside}

\end{pages} 
\Pages

\clearpage
\thispagestyle{empty}
\null\vfill
\settowidth\longest{\huge\itshape […] and when I turned around I saw}
\begin{center}
\parbox{\longest}{%
  \raggedright{\huge\itshape%
    ``[…] and when I turned around I saw seven golden lamp-stands; and in the midst of the lamp-stands was someone like the Son of Man.'' \par\bigskip
  }
  \raggedleft\Large\MakeUppercase{``Menschensohn'' — Gebhard Fugel, 1933}\par%
}
\vfill\vfill
\clearpage\newpage
\end{center}
\newpage
\thispagestyle{empty}
\begin{center}
	%\includegraphics[width=0.98\textwidth]{}
\end{center}
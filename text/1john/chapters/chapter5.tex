\begin{pages}
    \begin{Rightside}
    \selectlanguage{greek}
        \beginnumbering
		\pstart[
				\chapter{Ἰωάννης ἐν τῇ Πάτμῳ}
				\markboth{John on the Isle of Patmos}
				]
		\renewcommand{\LettrineFontHook}{\PHtitl}
		\lettrine[lines=3]{Π}{ᾶς} ὁ πιστεύων ὅτι Ἰησοῦς ἐστιν ὁ Χριστὸς ἐκ τοῦ Θεοῦ γεγέννηται, καὶ πᾶς ὁ ἀγαπῶν τὸν γεννήσαντα ἀγαπᾷ τὸν γεγεννημένον ἐξ αὐτοῦ. ἐν τούτῳ γινώσκομεν ὅτι ἀγαπῶμεν τὰ τέκνα τοῦ Θεοῦ, ὅταν τὸν Θεὸν ἀγαπῶμεν καὶ τὰς ἐντολὰς αὐτοῦ ποιῶμεν. αὕτη γάρ ἐστιν ἡ ἀγάπη τοῦ Θεοῦ, ἵνα τὰς ἐντολὰς αὐτοῦ τηρῶμεν· καὶ αἱ ἐντολαὶ αὐτοῦ βαρεῖαι οὐκ εἰσίν, ὅτι πᾶν τὸ γεγεννημένον ἐκ τοῦ Θεοῦ νικᾷ τὸν κόσμον· καὶ αὕτη ἐστὶν ἡ νίκη ἡ νικήσασα τὸν κόσμον, ἡ πίστις ἡμῶν. τίς ἐστιν ὁ νικῶν τὸν κόσμον εἰ μὴ ὁ πιστεύων ὅτι Ἰησοῦς ἐστιν ὁ Υἱὸς τοῦ Θεοῦ; οὗτός ἐστιν ὁ ἐλθὼν δι’ ὕδατος καὶ αἵματος, Ἰησοῦς Χριστός· οὐκ ἐν τῷ ὕδατι μόνον, ἀλλ’ ἐν τῷ ὕδατι καὶ ἐν τῷ αἵματι· καὶ τὸ Πνεῦμά ἐστιν τὸ μαρτυροῦν, ὅτι τὸ Πνεῦμά ἐστιν ἡ ἀλήθεια. ὅτι τρεῖς εἰσιν οἱ μαρτυροῦντες, τὸ Πνεῦμα καὶ τὸ ὕδωρ καὶ τὸ αἷμα, καὶ οἱ τρεῖς εἰς τὸ ἕν εἰσιν. εἰ τὴν μαρτυρίαν τῶν ἀνθρώπων λαμβάνομεν, ἡ μαρτυρία τοῦ Θεοῦ μείζων ἐστίν, ὅτι αὕτη ἐστὶν ἡ μαρτυρία τοῦ Θεοῦ ὅτι μεμαρτύρηκεν περὶ τοῦ Υἱοῦ αὐτοῦ. ὁ πιστεύων εἰς τὸν Υἱὸν τοῦ Θεοῦ ἔχει τὴν μαρτυρίαν ἐν αὑτῷ. ὁ μὴ πιστεύων τῷ Θεῷ ψεύστην πεποίηκεν αὐτόν, ὅτι οὐ πεπίστευκεν εἰς τὴν μαρτυρίαν ἣν μεμαρτύρηκεν ὁ Θεὸς περὶ τοῦ Υἱοῦ αὐτοῦ. καὶ αὕτη ἐστὶν ἡ μαρτυρία, ὅτι ζωὴν αἰώνιον ἔδωκεν ὁ Θεὸς ἡμῖν, καὶ αὕτη ἡ ζωὴ ἐν τῷ Υἱῷ αὐτοῦ ἐστιν. ὁ ἔχων τὸν Υἱὸν ἔχει τὴν ζωήν· ὁ μὴ ἔχων τὸν Υἱὸν τοῦ Θεοῦ τὴν ζωὴν οὐκ ἔχει. Ταῦτα ἔγραψα ὑμῖν ἵνα εἰδῆτε ὅτι ζωὴν ἔχετε αἰώνιον, τοῖς πιστεύουσιν εἰς τὸ ὄνομα τοῦ Υἱοῦ τοῦ Θεοῦ. Καὶ αὕτη ἐστὶν ἡ παρρησία ἣν ἔχομεν πρὸς αὐτόν, ὅτι ἐάν τι αἰτώμεθα κατὰ τὸ θέλημα αὐτοῦ ἀκούει ἡμῶν. καὶ ἐὰν οἴδαμεν ὅτι ἀκούει ἡμῶν ὃ ἐὰν αἰτώμεθα, οἴδαμεν ὅτι ἔχομεν τὰ αἰτήματα ἃ ᾐτήκαμεν ἀπ’ αὐτοῦ. Ἐάν τις ἴδῃ τὸν ἀδελφὸν αὐτοῦ ἁμαρτάνοντα ἁμαρτίαν μὴ πρὸς θάνατον, αἰτήσει, καὶ δώσει αὐτῷ ζωήν, τοῖς ἁμαρτάνουσιν μὴ πρὸς θάνατον. ἔστιν ἁμαρτία πρὸς θάνατον· οὐ περὶ ἐκείνης λέγω ἵνα ἐρωτήσῃ. πᾶσα ἀδικία ἁμαρτία ἐστίν, καὶ ἔστιν ἁμαρτία οὐ πρὸς θάνατον. Οἴδαμεν ὅτι πᾶς ὁ γεγεννημένος ἐκ τοῦ Θεοῦ οὐχ ἁμαρτάνει, ἀλλ’ ὁ γεννηθεὶς ἐκ τοῦ Θεοῦ τηρεῖ αὐτόν, καὶ ὁ πονηρὸς οὐχ ἅπτεται αὐτοῦ. οἴδαμεν ὅτι ἐκ τοῦ Θεοῦ ἐσμεν, καὶ ὁ κόσμος ὅλος ἐν τῷ πονηρῷ κεῖται. οἴδαμεν δὲ ὅτι ὁ Υἱὸς τοῦ Θεοῦ ἥκει, καὶ δέδωκεν ἡμῖν διάνοιαν ἵνα γινώσκομεν τὸν ἀληθινόν· καὶ ἐσμὲν ἐν τῷ ἀληθινῷ, ἐν τῷ Υἱῷ αὐτοῦ Ἰησοῦ Χριστῷ. οὗτός ἐστιν ὁ ἀληθινὸς Θεὸς καὶ ζωὴ αἰώνιος. Τεκνία, φυλάξατε ἑαυτὰ ἀπὸ τῶν εἰδώλων.
		\pend
        \endnumbering
    \end{Rightside}
    \begin{Leftside}
        \beginnumbering
        \pstart[
        			\chapter{John on the Isle of Patmos}
        			]
        	\renewcommand\LettrineFontHook{\Zallmanfamily}
		\lettrine[lines=3]{T}{he} Revelation of Jesus Christ, which God gave Him to show His servants what must soon happen; and He made it known through the sending of His messenger to His servant John, who confirms everything that he saw, namely the word of God and the testimony of Jesus Christ. Blessed is the reader and the people who listen to the words of the prophecy and (blessed is) the one who heeds what is written in it (the prophecy), for the time is near.
		\pend
        \endnumbering
    \end{Leftside}

\end{pages} 
\Pages

\clearpage
\thispagestyle{empty}
\null\vfill
\settowidth\longest{\huge\itshape […] and when I turned around I saw}
\begin{center}
\parbox{\longest}{%
  \raggedright{\huge\itshape%
    ``[…] and when I turned around I saw seven golden lamp-stands; and in the midst of the lamp-stands was someone like the Son of Man.'' \par\bigskip
  }
  \raggedleft\Large\MakeUppercase{``Menschensohn'' — Gebhard Fugel, 1933}\par%
}
\vfill\vfill
\clearpage\newpage
\end{center}
\newpage
\thispagestyle{empty}
\begin{center}
	%\includegraphics[width=0.98\textwidth]{}
\end{center}
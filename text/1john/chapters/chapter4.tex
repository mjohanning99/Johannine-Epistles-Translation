\begin{pages}
    \begin{Rightside}
    \selectlanguage{greek}
        \beginnumbering
		\pstart[
				\chapter{Ἰωάννης ἐν τῇ Πάτμῳ}
				\markboth{John on the Isle of Patmos}
				]
		\renewcommand{\LettrineFontHook}{\PHtitl}
		\lettrine[lines=3]{Ἀ}{γαπητοί}, μὴ παντὶ πνεύματι πιστεύετε, ἀλλὰ δοκιμάζετε τὰ πνεύματα εἰ ἐκ τοῦ Θεοῦ ἐστιν, ὅτι πολλοὶ ψευδοπροφῆται ἐξεληλύθασιν εἰς τὸν κόσμον. Ἐν τούτῳ γινώσκετε τὸ Πνεῦμα τοῦ Θεοῦ· πᾶν πνεῦμα ὃ ὁμολογεῖ Ἰησοῦν Χριστὸν ἐν σαρκὶ ἐληλυθότα ἐκ τοῦ Θεοῦ ἐστιν, καὶ πᾶν πνεῦμα ὃ μὴ ὁμολογεῖ τὸν Ἰησοῦν ἐκ τοῦ Θεοῦ οὐκ ἔστιν· καὶ τοῦτό ἐστιν τὸ τοῦ ἀντιχρίστου, ὃ ἀκηκόατε ὅτι ἔρχεται, καὶ νῦν ἐν τῷ κόσμῳ ἐστὶν ἤδη. Ὑμεῖς ἐκ τοῦ Θεοῦ ἐστε, τεκνία, καὶ νενικήκατε αὐτούς, ὅτι μείζων ἐστὶν ὁ ἐν ὑμῖν ἢ ὁ ἐν τῷ κόσμῳ. αὐτοὶ ἐκ τοῦ κόσμου εἰσίν· διὰ τοῦτο ἐκ τοῦ κόσμου λαλοῦσιν καὶ ὁ κόσμος αὐτῶν ἀκούει. ἡμεῖς ἐκ τοῦ Θεοῦ ἐσμεν· ὁ γινώσκων τὸν Θεὸν ἀκούει ἡμῶν, ὃς οὐκ ἔστιν ἐκ τοῦ Θεοῦ οὐκ ἀκούει ἡμῶν. ἐκ τούτου γινώσκομεν τὸ πνεῦμα τῆς ἀληθείας καὶ τὸ πνεῦμα τῆς πλάνης.
		\pend
		\pstart
	 	Ἀγαπητοί, ἀγαπῶμεν ἀλλήλους, ὅτι ἡ ἀγάπη ἐκ τοῦ Θεοῦ ἐστιν, καὶ πᾶς ὁ ἀγαπῶν ἐκ τοῦ Θεοῦ γεγέννηται καὶ γινώσκει τὸν Θεόν. ὁ μὴ ἀγαπῶν οὐκ ἔγνω τὸν Θεόν, ὅτι ὁ Θεὸς ἀγάπη ἐστίν. ἐν τούτῳ ἐφανερώθη ἡ ἀγάπη τοῦ Θεοῦ ἐν ἡμῖν, ὅτι τὸν Υἱὸν αὐτοῦ τὸν μονογενῆ ἀπέσταλκεν ὁ Θεὸς εἰς τὸν κόσμον ἵνα ζήσωμεν δι’ αὐτοῦ. ἐν τούτῳ ἐστὶν ἡ ἀγάπη, οὐχ ὅτι ἡμεῖς ἠγαπήκαμεν τὸν Θεόν, ἀλλ’ ὅτι αὐτὸς ἠγάπησεν ἡμᾶς καὶ ἀπέστειλεν τὸν Υἱὸν αὐτοῦ ἱλασμὸν περὶ τῶν ἁμαρτιῶν ἡμῶν. Ἀγαπητοί, εἰ οὕτως ὁ Θεὸς ἠγάπησεν ἡμᾶς, καὶ ἡμεῖς ὀφείλομεν ἀλλήλους ἀγαπᾶν. Θεὸν οὐδεὶς πώποτε τεθέαται· ἐὰν ἀγαπῶμεν ἀλλήλους, ὁ Θεὸς ἐν ἡμῖν μένει καὶ ἡ ἀγάπη αὐτοῦ τετελειωμένη ἐν ἡμῖν ἐστιν. ἐν τούτῳ γινώσκομεν ὅτι ἐν αὐτῷ μένομεν καὶ αὐτὸς ἐν ἡμῖν, ὅτι ἐκ τοῦ Πνεύματος αὐτοῦ δέδωκεν ἡμῖν. καὶ ἡμεῖς τεθεάμεθα καὶ μαρτυροῦμεν ὅτι ὁ Πατὴρ ἀπέσταλκεν τὸν Υἱὸν Σωτῆρα τοῦ κόσμου. ὃς ἐὰν ὁμολογήσῃ ὅτι Ἰησοῦς ἐστιν ὁ Υἱὸς τοῦ Θεοῦ, ὁ Θεὸς ἐν αὐτῷ μένει καὶ αὐτὸς ἐν τῷ Θεῷ. καὶ ἡμεῖς ἐγνώκαμεν καὶ πεπιστεύκαμεν τὴν ἀγάπην ἣν ἔχει ὁ Θεὸς ἐν ἡμῖν. Ὁ Θεὸς ἀγάπη ἐστίν, καὶ ὁ μένων ἐν τῇ ἀγάπῃ ἐν τῷ Θεῷ μένει καὶ ὁ Θεὸς ἐν αὐτῷ μένει. Ἐν τούτῳ τετελείωται ἡ ἀγάπη μεθ’ ἡμῶν, ἵνα παρρησίαν ἔχωμεν ἐν τῇ ἡμέρᾳ τῆς κρίσεως, ὅτι καθὼς ἐκεῖνός ἐστιν καὶ ἡμεῖς ἐσμεν ἐν τῷ κόσμῳ τούτῳ. φόβος οὐκ ἔστιν ἐν τῇ ἀγάπῃ, ἀλλ’ ἡ τελεία ἀγάπη ἔξω βάλλει τὸν φόβον, ὅτι ὁ φόβος κόλασιν ἔχει, ὁ δὲ φοβούμενος οὐ τετελείωται ἐν τῇ ἀγάπῃ. Ἡμεῖς ἀγαπῶμεν, ὅτι αὐτὸς πρῶτος ἠγάπησεν ἡμᾶς. ἐάν τις εἴπῃ ὅτι Ἀγαπῶ τὸν Θεόν, καὶ τὸν ἀδελφὸν αὐτοῦ μισῇ, ψεύστης ἐστίν· ὁ γὰρ μὴ ἀγαπῶν τὸν ἀδελφὸν αὐτοῦ ὃν ἑώρακεν, τὸν Θεὸν ὃν οὐχ ἑώρακεν οὐ δύναται ἀγαπᾶν. καὶ ταύτην τὴν ἐντολὴν ἔχομεν ἀπ’ αὐτοῦ, ἵνα ὁ ἀγαπῶν τὸν Θεὸν ἀγαπᾷ καὶ τὸν ἀδελφὸν αὐτοῦ.
		\pend
        \endnumbering
    \end{Rightside}
    \begin{Leftside}
        \beginnumbering
        \pstart[
        			\chapter{John on the Isle of Patmos}
        			]
        	\renewcommand\LettrineFontHook{\Zallmanfamily}
		\lettrine[lines=3]{T}{he} Revelation of Jesus Christ, which God gave Him to show His servants what must soon happen; and He made it known through the sending of His messenger to His servant John, who confirms everything that he saw, namely the word of God and the testimony of Jesus Christ. Blessed is the reader and the people who listen to the words of the prophecy and (blessed is) the one who heeds what is written in it (the prophecy), for the time is near.
		\pend
        \endnumbering
    \end{Leftside}

\end{pages} 
\Pages

\clearpage
\thispagestyle{empty}
\null\vfill
\settowidth\longest{\huge\itshape […] and when I turned around I saw}
\begin{center}
\parbox{\longest}{%
  \raggedright{\huge\itshape%
    ``[…] and when I turned around I saw seven golden lamp-stands; and in the midst of the lamp-stands was someone like the Son of Man.'' \par\bigskip
  }
  \raggedleft\Large\MakeUppercase{``Menschensohn'' — Gebhard Fugel, 1933}\par%
}
\vfill\vfill
\clearpage\newpage
\end{center}
\newpage
\thispagestyle{empty}
\begin{center}
	%\includegraphics[width=0.98\textwidth]{}
\end{center}